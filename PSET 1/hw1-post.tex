\documentclass[11pt]{article}
\usepackage[dvips]{color}
\usepackage{epsf}
\usepackage{graphicx}
% \usepackage{graphics}
\usepackage{palatino}
\usepackage{amsmath,amssymb}
\parindent=0pt
\parskip=11pt
% \pagestyle{empty}
\setlength{\jot}{4pt}
\setlength{\textwidth}{6.5in}
\setlength{\textheight}{9.0in}
\setlength{\evensidemargin}{0.0in}
\setlength{\oddsidemargin}{0.0in}
\setlength{\topmargin}{-.5in}
\newcommand{\newc}{\newcommand}
\newc{\bc}{\begin{center}}
\newc{\ec}{\end{center}}
\newc{\be}{\begin{enumerate}}
\newc{\ee}{\end{enumerate}}
\newc{\bi}{\begin{itemize}}
\newc{\ei}{\end{itemize}}
\newc{\bd}{\begin{description}}
\newc{\ed}{\end{description}}
\newc{\und}[1]{\underline{#1}}
\newc{\E}{\mbox{E}}
\newc{\V}{\mbox{V}}
\newc{\N}{\mbox{N}}
\newc{\B}{\mbox{Bin}}
\newc{\Bern}{\mbox{Bern}}
\newc{\Po}{\mbox{Po}}
\newc{\IG}{\mbox{IG}}
\newc{\Gam}{\mbox{Gam}}
% \newc{\bdP}{\mbox{P}}
\newc{\bdp}{\mathsf{p}}
\newc{\odds}{\mbox{odds}}
\newc{\OR}{\mbox{OR}}
\newc{\stderr}{\mbox{s.e.}}
\newc{\logit}{\mbox{logit}}
\newc{\sign}{\mbox{sign}}
\newc{\SD}{\mbox{SD}}
\newc{\bdmu}{\mbox{\boldmath $\mu$}}
\newc{\bdSigma}{\mbox{\boldmath $\Sigma$}}
\newc{\bdLambda}{\mbox{\boldmath $\Lambda$}}
\newc{\bdmuhat}{\mbox{\boldmath $\hat{\mu}$}}
\newc{\bdeta}{\mbox{\boldmath $\eta$}}
\newc{\bdtheta}{\mbox{\boldmath $\theta$}}
\newc{\bdbeta}{\mbox{\boldmath $\beta$}}
\newc{\bdgamma}{\mbox{\boldmath $\gamma$}}
\newc{\bdbetahat}{\mbox{\boldmath $\hat{\beta}$}}
\newc{\bdgammahat}{\mbox{\boldmath $\hat{\gamma}$}}
\newc{\bdthetahat}{\mbox{\boldmath $\hat{\theta}$}}
\newc{\bdvareps}{\mbox{\boldmath $\varepsilon$}}
\newc{\bdzero}{\mbox{\boldmath $0$}}
\newc{\bdone}{\mbox{\boldmath $1$}}
\newc{\bdnu}{\mbox{\boldmath $\nu$}}
\newc{\bdell}{\mbox{\boldmath $\ell$}}
\newc{\bdxi}{\mbox{\boldmath $\xi$}}
\newc{\bdomega}{\mbox{\boldmath $\omega$}}
\newc{\bdepsilon}{\mbox{\boldmath $\varepsilon$}}
\newc{\bdI}{\mathbf{I}}
\newc{\bdP}{\mbox{\boldmath $P$}}
\newc{\bdX}{\mbox{\boldmath $X$}}
\newc{\bdA}{\mbox{\boldmath $A$}}
\newc{\bdB}{\mbox{\boldmath $B$}}
\newc{\bdC}{\mbox{\boldmath $C$}}
\newc{\bdD}{\mbox{\boldmath $D$}}
\newc{\bdG}{\mbox{\boldmath $G$}}
\newc{\bdJ}{\mbox{\boldmath $J$}}
\newc{\bdK}{\mbox{\boldmath $K$}}
\newc{\bda}{\mbox{\boldmath $a$}}
\newc{\bdb}{\mbox{\boldmath $b$}}
\newc{\bdc}{\mbox{\boldmath $c$}}
\newc{\bde}{\mbox{\boldmath $e$}}
\newc{\bdu}{\mbox{\boldmath $u$}}
\newc{\bdv}{\mbox{\boldmath $v$}}
\newc{\bdx}{\mbox{\boldmath $x$}}
\newc{\bdy}{\mbox{\boldmath $y$}}
\newc{\bdz}{\mbox{\boldmath $z$}}
\newc{\bdr}{\mbox{\boldmath $r$}}
\newc{\bdQ}{\mbox{\boldmath $Q$}}
\newc{\bdR}{\mbox{\boldmath $R$}}
\newc{\bdY}{\mbox{\boldmath $Y$}}
\newc{\bdT}{\mbox{\boldmath $T$}}
\newc{\bdW}{\mbox{\boldmath $W$}}
\newc{\bdH}{\mbox{\boldmath $H$}}
\newc{\bdL}{\mbox{\boldmath $L$}}
\newc{\bdU}{\mbox{\boldmath $U$}}
\newc{\bdV}{\mbox{\boldmath $V$}}
\newc{\Multinom}{\mbox{Multinom}}
\newc{\Var}{\mbox{Var}}
\newc{\var}{\mbox{var}}
\newc{\diag}{\mbox{diag}}
\newc{\tr}{\mbox{tr}}
\newc{\phat}{\hat{p}}
\newc{\Xbar}{\bar{X}}
\newc{\xbar}{\bar{x}}
\newc{\Ybar}{\bar{Y}}
\newc{\ybar}{\bar{y}}
\newc{\dbar}{\bar{d}}
\newc{\yhat}{\hat{y}}
\newc{\bdyhat}{\mbox{\boldmath $\hat{y}$}}
\newc{\ytil}{\tilde{y}}
\newc{\ftil}{\tilde{f}}
\newc{\Ho}{\mbox{\bf H}_o}
\newc{\Ha}{\mbox{\bf H}_a}
\newc{\phatYX}{\phat_Y - \phat_X}
\newc{\SSG}{\mbox{SSG}}
\newc{\SSB}{\mbox{SSB}}
\newc{\SSE}{\mbox{SSE}}
\newc{\SST}{\mbox{SST}}
\newc{\SSR}{\mbox{SSR}}
\newc{\SSAB}{\mbox{SSAB}}
\newc{\MSG}{\mbox{MSG}}
\newc{\MSB}{\mbox{MSB}}
\newc{\MSE}{\mbox{MSE}}
\newc{\MST}{\mbox{MST}}
\newc{\MSAB}{\mbox{MSAB}}
\newc{\dfE}{\mbox{dfE}}
\newc{\dfG}{\mbox{dfG}}
\newc{\dfB}{\mbox{dfB}}
\newc{\dfT}{\mbox{dfT}}
\newc{\dfAB}{\mbox{dfAB}}
\newc{\muhat}{\hat{\mu}}
\newc{\betahat}{\hat{\beta}}
\newc{\alphahat}{\hat{\alpha}}
\newc{\etahat}{\hat{\eta}}
\newc{\phihat}{\hat{\phi}}
\newc{\sigmahat}{\hat{\sigma}}
\newc{\cl}{\centerline}
\newc{\redtitle}[1]{ {\color{red}\und{#1}:} }
\newc{\bluetitle}[1]{ {\color{blue}\und{#1}:} }
\newc{\magentatitle}[1]{ {\color{magenta}\und{#1}:} }
\newc{\R}{\mathbb{R}}
\newc{\trans}{^\mathsf{T}}
\newc{\xtx}{\bdX\trans\bdX}
\newc{\xxtxx}{\bdX(\xtx)^{-1}\bdX\trans}
\newc{\argmin}{\operatornamewithlimits{argmin}}
\newc{\argmax}{\operatornamewithlimits{argmax}}
\newc{\bs}{}
\newc{\es}{}
\newc{\sm}{}
\newc{\tny}{\small}
\newc{\bpage}{\eject}
\newc{\incg}[1]{
\vspace*{0.2in}
\includegraphics[scale=0.55,angle=270]{#1}
}
% \flushbottom

\pagestyle{empty}
\usepackage{tcolorbox}
\newtcolorbox{hwanswer}{
    colback=gray!15,          % Light gray background
    colframe=gray!50,         % Slightly darker gray border
    arc=6pt,                  % Rounded corners (6pt radius)
    boxrule=1pt,              % Border thickness
    left=8pt,                 % Left padding
    right=8pt,                % Right padding
    top=8pt,                  % Top padding
    bottom=8pt,               % Bottom padding
    breakable,                % Allow box to break across pages
    enhanced                  % Enable advanced features
}
\usepackage{mathrsfs}
\usepackage{bm}

\begin{document}

\begin{center}
{\large\bf Statistics 244 --- Fall 2025 --- Assignment 1}\\
Due
Friday, 
September 19, 2025
\end{center}

Homework is to be 
uploaded to Gradescope by 10:00pm on Friday evening.

\underline{Readings}:

``Introductory lecture''  and
``Linear algebra issues in linear models'' 
course notes.

Agresti:  Chapter 1 

\underline{Written assignment}

\begin{enumerate}
\item Suppose $\bdX$ is an $n\times p$ model matrix.
Show that
$\{\bdX\bdbeta | \bdbeta \in \R^p \}$ 
is a vector space.

\begin{hwanswer}
\textbf{Solution:}
	Let $V = \{\bdX\bdbeta | \bdbeta \in \R^p \}$.  We simply need to show that that set of vectors is closed under linear combination, i.e a) $\bm{u}+\bm v \in V$ and $c \bm u \in V$. Since we are looking for $\beta \in \mathbb{R}^p$, we are simply talking about any arbitrary linear combination of the columns of $\bdX$. Let us first prove the first property: 
	\be
	\item if $\bm u \in V$ and $\bm v \in V$, $\bm u + \bm v \in V$. By the definition of $V$, $\exists \beta_1, \beta_2$ such that $\bdX \beta_1 = \bm u$ and $\bdX \beta_2 = \bm v$. In that case, we can rewrite their sum as $\bdX (\beta_1 + \beta_2)$. By the definition of $\beta$, as $\beta_1 + \beta_2 \in \mathbb R^p$, $\bm u + \bm v \in V$.
	\item if $\bm u \in V$, then $c \bm u \in V$. This is even simpler to show. By definition of $V$, $\exists \beta_3$ such that $\bdX \beta_3=\bm u$. Then, $c \bm u = c \bdX \beta_3 = \bdX c\beta_3$, and as $c$ is a real scalar, $c \beta_3 \in \mathbb R^p$.
	\ee 
	Thus, $\{\bdX\bdbeta | \bdbeta \in \R^p \}$ is a vector space. 	
\end{hwanswer}


\item 
\be
\item For $n\times p$ model matrix $\bdX$, show that the null space of
$\bdX$, $N(\bdX)$, is the orthogonal complement of the column space of
$\bdX\trans$, that is, $C(\bdX\trans)^\perp$. 
\begin{hwanswer}
	\textbf{Solution:} The definition of orthogonal complement is 
\begin{align*}
	\forall \bm u \in W, \forall \bm v \in W^\bot \subseteq \mathbb{R}^n \iff \bm u^\top \bm v = 0
\end{align*}
Here, if the equation on the right is satisfied, then $W^\bot$ is the orthogonal complement of $W$, and vice versa.

As for null space, it is defined as
\begin{align*}
	N(\bdX)=\{\bm \zeta : \bdX \bm \zeta = \bm 0\}
\end{align*}
This means that the null space of $\bdX$ is simply the set of vectors for which $\bm \zeta$ is orthogonal to all rows in $\bdX$, as their multiplication should produce the $\bm 0$, i.e the zero vector. 

Let $\bm k^\top$ and $\bm \ell^\top$ be two rows from $\bdX$. We want to show being orthogonal to each separately, i.e $\bm k^\top \bm \zeta =\bm 0$ and $\bm \ell^\top \bm \zeta =\bm 0$, means being orthogonal to any arbitrary linear combination too. This follows trivially: 

\begin{align*}
	(c \bm k^\top+p\bm \ell^\top)\bm \zeta = c\bm k^\top \bm \zeta + p \bm \ell ^\top \bm \zeta = c \cdot 0 + p \cdot 0 = 0
\end{align*}

Then, it can be said that $N(\bdX)$ is orthogonal to all linear combinations of the rows of $\bdX$. As the column space of $\bdX^\top$ is simply the vector space created by the span of the columns of $\bdX^\top$, and the span of the columns of $\bdX^\top$ is equivalent to the span of the rows of $\bdX$, $N(\bdX) = C(X^\top)^\bot$.
\end{hwanswer}
\item Let $\bdX =
\left( \begin{array}{rl} 1 & 0 \\ 1 & 1 \\ 0 & 1 \end{array} \right)$,
and let $V = C(\bdX)$ be the vector space spanned by the columns of
$\bdX$.
Determine the orthogonal complement $V^\perp$ of $V$ as a closed-form
expression.
Use the result of part (a) as a guide.\\
\begin{hwanswer}
	\textbf{Solution:} We saw in part (a) that the null space of $\bdX$ is equivalent to the orthogonal complement of $C(X^\top)$. Here, we want to look at it from the other way. The orthogonal complement $V^\bot$ of $V$ can be written as the null space of the transpose of the model matrix that creates it, i.e 
\begin{align*}
	V^\bot = N(\bm X^\top)
\end{align*}
Considering the simplicity of $\bdX$, this is fairly simple: 
\begin{align*}
	N(\bdX^\top) = \{\bm \zeta : \bdX^\top \bm \zeta = \bm 0\} = \{\bm \zeta : \bm \zeta = c (-1,1,-1)^\top \forall c \in \mathbb R\} = \text{span}((-1,1,-1)^\top)
\end{align*}
\end{hwanswer}
\ee

\item 
A model $M$ has model matrix $\bdX$. 
A simpler model $M_0$ results from removing the final term in $M$, 
and hence has model matrix $\bdX_0$ that deletes the final column from
$\bdX$. 
From the definition of a column space, explain why 
$C(\bdX_0) \subseteq C(\bdX)$.

\begin{hwanswer}
	\textbf{Solution:} Looking back at the definition of column space, we see 

\begin{align*}
	C(\bdX) = \{\bm \eta : \bm \eta = \bdX \beta, \forall \beta \in \mathbb R^p\} \subseteq \mathbb R^n
\end{align*}

We want to show that $C(\bdX_0)$ is either a subset of or equal to $C(\bdX)$. An important detail here to notice is that the model matrix $\bdX$ isn't said to be full-rank, which is where the possibility of $\bdX_0$ and $\bdX$ having the same column space comes from. Here, we have two options: 
a) Either a linearly independent vector was removed, in which case $\bdX_0$ is a subset, or b) a dependent vector was removed, in which case $\bdX_0$ is the same as $\bdX$. If we prove these two statements, then we will have proven the statement in the question. However, another way to approach this which is agnostic of whether $\bm x_p$ was linearly independent or not is showing that if something is a member of $C(\bdX_0)$, then it also has to be a member of $C(\bdX)$. First, we can rewrite the definition of the column space as such: 

\begin{align*}
	C(\bdX) = \{\bm \eta : \bm \eta = \bm x_1 \beta_1 + \bm x_2 \beta_2 + \dots + \bm x_p \beta_p, \forall \beta_i \in \mathbb R\} \subseteq \mathbb R^n
\end{align*}

Then, the column space of $\bdX_0$ is defined as:

\begin{align*}
	C(\bdX_0) = \{\bm \eta : \bm \eta = \bm x_1 \beta_1 + \bm x_2 \beta_2 + \dots + \bm x_{p-1} \beta_{p-1}, \forall \beta_i \in \mathbb R\} \subseteq \mathbb R^n
\end{align*}

It is fairly evident from this example that 
\begin{align*}
	C(\bdX_0) \subseteq C(\bdX)
\end{align*}

as when $\beta_p$ is zero, the set of vectors that can be expressed by both become equal, but $C(\bdX)$ is strictly larger when $\beta_p$ isn't zero. 

\end{hwanswer}
\item 
Suppose that $\bdX_1$ and $\bdX_2$ are full-rank 
$n\times p$ model matrices with $p<n$.
\be
\item Show that if 
$\bdA \in \R^{p\times p}$ 
is nonsingular, 
$C(\bdX_1) = C(\bdX_1\bdA)$.
\item Now suppose that $C(\bdX_1) = C(\bdX_2)$.
This part of the problem involves
showing that this implies the existence of a nonsingular matrix
$\bdA \in \R^{p\times p}$ 
such that $\bdX_1 = \bdX_2 \bdA$.
\be
\item To establish the existence of $\bdA$, let 
\[
\bdX_1 = [\bdx_{11} \mid \bdx_{12} \mid \cdots \mid \bdx_{1p}],
\hspace{0.5in}
\bdA = [\bda_1 \mid \cdots \mid \bda_p],
\]
where $\bdx_{1j}$ is the $j$-th column of $\bdX_1$,
and $\bda_j \in \R^p$ is the $j$-th column of $\bdA$.
Show that each
$\bdx_{1j}$ can be written as 
$\bdx_{1j} = \bdX_2 \bda_j$, and hence $\bdX_1 = \bdX_2\bdA$.
\item With the result from part~(i), 
show that $\bdA$ must be invertible.
Prove this by contradiction:
assume that $\bdA$ is
singular, so there exists a nonzero vector $\bdv \in \R^p$ 
such that $\bdA\bdv = \bdzero$.
Proceed by showing that this contradicts the assumption that $\bdX_1$
has full column rank.
\ee
\ee

\item 
Consider the model for the \textit{two-way layout} for 
categorical variables
$A$ and $B$:
\[
\E(y_{ijk}) = \beta_0 + \beta_i + \gamma_j, \hspace{0.1 in}
\text{where } i = 1, \dots, r; \hspace{0.1 in} j = 1, \dots,
c;\hspace{0.1 in} \text{ and } k = 1, \dots, n.
\]
This model is \textit{balanced}, having an equal sample size $n$ in each 
of the $rc$ cells. 
We might be interested in estimating the parameter vector
$(\beta_0, \beta_1,\ldots, \beta_r, \gamma_1,\ldots,\gamma_c)$.
\be
\item For the model as stated, is the parameter vector identifiable? 
Why or why not?
It might help to code the model for $\E(y_{ijk})$ into a model matrix 
that multiplies the parameter vector stated above.
\item Give an example of a quantity that is a function of
the model parameters,
$\beta_0, \beta_1,\ldots, \beta_r, \gamma_1,\ldots,\gamma_c$,
that is (i) not estimable, and (ii) estimable. 
In each case, provide a rigorous justification.
\ee

\item 
Suppose $\bdy \in \R^n$, and $\E(\bdy) = \bdX\bdbeta$ for
$n\times p$ model matrix $\bdX$ and $\bdbeta \in \R^p$.
This question concerns proving
that $\bdell\trans\bdbeta$ is estimable if and only if
\[
\bdell\trans = \bdell\trans (\bdX\trans\bdX)^{-}(\bdX\trans\bdX) 
\]
for selected $\bdell \in \R^p$.
\be
\item 
Prove that if $\bdell\trans\bdbeta$ is estimable, then
for any g-inverse 
$(\bdX\trans\bdX)^{-}$
of $\bdX\trans\bdX$,
\[
\bdell\trans = \bdell\trans (\bdX\trans\bdX)^{-}(\bdX\trans\bdX) 
\]
for selected $\bdell \in \R^p$.

\item
Prove that if 
\[
\bdell\trans = \bdell\trans (\bdX\trans\bdX)^{-}(\bdX\trans\bdX) 
\]
for selected $\bdell \in \R^p$ and for some g-inverse
$(\bdX\trans\bdX)^{-}$
of $\bdX\trans\bdX$, then
$\bdell\trans\bdbeta$ is estimable.

\item
For full-rank $\bdX$, what do the previous results imply about the choice
of $\bdell \in \R^p$ for $\bdell\trans\bdbeta$ to be estimable?
\ee

\item 
Suppose $\bdW_1$ and $\bdW_2$ are two arbitrary subspaces of $\R^n$.
The goal of this problem is to prove
\[
(\bdW_1 \cap \bdW_2)^\perp
=
\bdW_1^\perp + \bdW_2^\perp,
\]
where, in general, for $V_1, V_2 \subseteq \R^n$,
\[
V_1 + V_2 = \{v_1 + v_2 : v_1\in V_1, v_2\in V_2\}.
\]
You may assume below (without proof) that 
for any subspace $A \subseteq \R^n$, $(A^\perp)^\perp = A$.
\be
\item 
We first need to prove three lemmas.
\be
\item 
Prove that,
for any subspace $A \subseteq \R^n$, $(A^\perp)^\perp = A$.
\item 
Show that, for any $A \subseteq \R^n$, $\bdzero \in A^\perp$.
\item 
Let $A,B \subseteq \R^n$.
Prove that $A\subseteq B$ if and only if 
$B^\perp \subseteq A^\perp$.
\ee
\item 
To prove the main result of the problem, we first want to show that 
\[
(\bdW_1 \cap \bdW_2)^\perp
\subseteq
\bdW_1^\perp + \bdW_2^\perp.
\]
\be
\item 
Show that $\bdW_i^\perp \subseteq \bdW_1^\perp + \bdW_2^\perp$ for $i=1,2$.
\item 
Show that the above implies 
\[
(\bdW_1^\perp + \bdW_2^\perp)^\perp \subseteq (\bdW_1 \cap\bdW_2),
\]
and therefore $(\bdW_1 \cap \bdW_2)^\perp
\subseteq
\bdW_1^\perp + \bdW_2^\perp$.
\ee
\item 
Now show the converse of part~(b), namely that
\[
\bdW_1^\perp + \bdW_2^\perp
\subseteq
(\bdW_1 \cap \bdW_2)^\perp.
\]
{\em Hint:  Choose
$v_1 \in \bdW_1^\perp$ and $v_2 \in \bdW_2^\perp$,
and proceed from there.}
\ee

\end{enumerate}
\end{document}
